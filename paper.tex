
%% bare_conf.tex
%% V1.4b
%% 2015/08/26
%% by Michael Shell
%% See:
%% http://www.michaelshell.org/
%% for current contact information.
%%
%% This is a skeleton file demonstrating the use of IEEEtran.cls
%% (requires IEEEtran.cls version 1.8b or later) with an IEEE
%% conference paper.
%%
%% Support sites:
%% http://www.michaelshell.org/tex/ieeetran/
%% http://www.ctan.org/pkg/ieeetran
%% and
%% http://www.ieee.org/

%%*************************************************************************
%% Legal Notice:
%% This code is offered as-is without any warranty either expressed or
%% implied; without even the implied warranty of MERCHANTABILITY or
%% FITNESS FOR A PARTICULAR PURPOSE! 
%% User assumes all risk.
%% In no event shall the IEEE or any contributor to this code be liable for
%% any damages or losses, including, but not limited to, incidental,
%% consequential, or any other damages, resulting from the use or misuse
%% of any information contained here.
%%
%% All comments are the opinions of their respective authors and are not
%% necessarily endorsed by the IEEE.
%%
%% This work is distributed under the LaTeX Project Public License (LPPL)
%% ( http://www.latex-project.org/ ) version 1.3, and may be freely used,
%% distributed and modified. A copy of the LPPL, version 1.3, is included
%% in the base LaTeX documentation of all distributions of LaTeX released
%% 2003/12/01 or later.
%% Retain all contribution notices and credits.
%% ** Modified files should be clearly indicated as such, including  **
%% ** renaming them and changing author support contact information. **
%%*************************************************************************


% *** Authors should verify (and, if needed, correct) their LaTeX system  ***
% *** with the testflow diagnostic prior to trusting their LaTeX platform ***
% *** with production work. The IEEE's font choices and paper sizes can   ***
% *** trigger bugs that do not appear when using other class files.       ***                          ***
% The testflow support page is at:
% http://www.michaelshell.org/tex/testflow/



\documentclass[conference]{IEEEtran}
% Some Computer Society conferences also require the compsoc mode option,
% but others use the standard conference format.
%
% If IEEEtran.cls has not been installed into the LaTeX system files,
% manually specify the path to it like:
% \documentclass[conference]{../sty/IEEEtran}





% Some very useful LaTeX packages include:
% (uncomment the ones you want to load)


% *** MISC UTILITY PACKAGES ***
%
%\usepackage{ifpdf}
% Heiko Oberdiek's ifpdf.sty is very useful if you need conditional
% compilation based on whether the output is pdf or dvi.
% usage:
% \ifpdf
%   % pdf code
% \else
%   % dvi code
% \fi
% The latest version of ifpdf.sty can be obtained from:
% http://www.ctan.org/pkg/ifpdf
% Also, note that IEEEtran.cls V1.7 and later provides a builtin
% \ifCLASSINFOpdf conditional that works the same way.
% When switching from latex to pdflatex and vice-versa, the compiler may
% have to be run twice to clear warning/error messages.






% *** CITATION PACKAGES ***
%
\usepackage{cite}
% cite.sty was written by Donald Arseneau
% V1.6 and later of IEEEtran pre-defines the format of the cite.sty package
% \cite{} output to follow that of the IEEE. Loading the cite package will
% result in citation numbers being automatically sorted and properly
% "compressed/ranged". e.g., [1], [9], [2], [7], [5], [6] without using
% cite.sty will become [1], [2], [5]--[7], [9] using cite.sty. cite.sty's
% \cite will automatically add leading space, if needed. Use cite.sty's
% noadjust option (cite.sty V3.8 and later) if you want to turn this off
% such as if a citation ever needs to be enclosed in parenthesis.
% cite.sty is already installed on most LaTeX systems. Be sure and use
% version 5.0 (2009-03-20) and later if using hyperref.sty.
% The latest version can be obtained at:
% http://www.ctan.org/pkg/cite
% The documentation is contained in the cite.sty file itself.






% *** GRAPHICS RELATED PACKAGES ***
%
\ifCLASSINFOpdf
 \usepackage[pdftex]{graphicx}
  % declare the path(s) where your graphic files are
  % \graphicspath{{../pdf/}{../jpeg/}}
  % and their extensions so you won't have to specify these with
  % every instance of \includegraphics
 \DeclareGraphicsExtensions{.pdf,.jpeg,.png}
\else
  % or other class option (dvipsone, dvipdf, if not using dvips). graphicx
  % will default to the driver specified in the system graphics.cfg if no
  % driver is specified.
 \usepackage[dvips]{graphicx}
  % declare the path(s) where your graphic files are
  % \graphicspath{{../eps/}}
  % and their extensions so you won't have to specify these with
  % every instance of \includegraphics
 \DeclareGraphicsExtensions{.eps}
\fi
% graphicx was written by David Carlisle and Sebastian Rahtz. It is
% required if you want graphics, photos, etc. graphicx.sty is already
% installed on most LaTeX systems. The latest version and documentation
% can be obtained at: 
% http://www.ctan.org/pkg/graphicx
% Another good source of documentation is "Using Imported Graphics in
% LaTeX2e" by Keith Reckdahl which can be found at:
% http://www.ctan.org/pkg/epslatex
%
% latex, and pdflatex in dvi mode, support graphics in encapsulated
% postscript (.eps) format. pdflatex in pdf mode supports graphics
% in .pdf, .jpeg, .png and .mps (metapost) formats. Users should ensure
% that all non-photo figures use a vector format (.eps, .pdf, .mps) and
% not a bitmapped formats (.jpeg, .png). The IEEE frowns on bitmapped formats
% which can result in "jaggedy"/blurry rendering of lines and letters as
% well as large increases in file sizes.
%
% You can find documentation about the pdfTeX application at:
% http://www.tug.org/applications/pdftex





% *** MATH PACKAGES ***
%
\usepackage{amsmath}
% A popular package from the American Mathematical Society that provides
% many useful and powerful commands for dealing with mathematics.
%
% Note that the amsmath package sets \interdisplaylinepenalty to 10000
% thus preventing page breaks from occurring within multiline equations. Use:
%\interdisplaylinepenalty=2500
% after loading amsmath to restore such page breaks as IEEEtran.cls normally
% does. amsmath.sty is already installed on most LaTeX systems. The latest
% version and documentation can be obtained at:
% http://www.ctan.org/pkg/amsmath





% *** SPECIALIZED LIST PACKAGES ***
%
%\usepackage{algorithmic}
% algorithmic.sty was written by Peter Williams and Rogerio Brito.
% This package provides an algorithmic environment fo describing algorithms.
% You can use the algorithmic environment in-text or within a figure
% environment to provide for a floating algorithm. Do NOT use the algorithm
% floating environment provided by algorithm.sty (by the same authors) or
% algorithm2e.sty (by Christophe Fiorio) as the IEEE does not use dedicated
% algorithm float types and packages that provide these will not provide
% correct IEEE style captions. The latest version and documentation of
% algorithmic.sty can be obtained at:
% http://www.ctan.org/pkg/algorithms
% Also of interest may be the (relatively newer and more customizable)
% algorithmicx.sty package by Szasz Janos:
% http://www.ctan.org/pkg/algorithmicx




% *** ALIGNMENT PACKAGES ***
%
%\usepackage{array}
% Frank Mittelbach's and David Carlisle's array.sty patches and improves
% the standard LaTeX2e array and tabular environments to provide better
% appearance and additional user controls. As the default LaTeX2e table
% generation code is lacking to the point of almost being broken with
% respect to the quality of the end results, all users are strongly
% advised to use an enhanced (at the very least that provided by array.sty)
% set of table tools. array.sty is already installed on most systems. The
% latest version and documentation can be obtained at:
% http://www.ctan.org/pkg/array


% IEEEtran contains the IEEEeqnarray family of commands that can be used to
% generate multiline equations as well as matrices, tables, etc., of high
% quality.




% *** SUBFIGURE PACKAGES ***
%\ifCLASSOPTIONcompsoc
%  \usepackage[caption=false,font=normalsize,labelfont=sf,textfont=sf]{subfig}
%\else
%  \usepackage[caption=false,font=footnotesize]{subfig}
%\fi
% subfig.sty, written by Steven Douglas Cochran, is the modern replacement
% for subfigure.sty, the latter of which is no longer maintained and is
% incompatible with some LaTeX packages including fixltx2e. However,
% subfig.sty requires and automatically loads Axel Sommerfeldt's caption.sty
% which will override IEEEtran.cls' handling of captions and this will result
% in non-IEEE style figure/table captions. To prevent this problem, be sure
% and invoke subfig.sty's "caption=false" package option (available since
% subfig.sty version 1.3, 2005/06/28) as this is will preserve IEEEtran.cls
% handling of captions.
% Note that the Computer Society format requires a larger sans serif font
% than the serif footnote size font used in traditional IEEE formatting
% and thus the need to invoke different subfig.sty package options depending
% on whether compsoc mode has been enabled.
%
% The latest version and documentation of subfig.sty can be obtained at:
% http://www.ctan.org/pkg/subfig




% *** FLOAT PACKAGES ***
%
%\usepackage{fixltx2e}
% fixltx2e, the successor to the earlier fix2col.sty, was written by
% Frank Mittelbach and David Carlisle. This package corrects a few problems
% in the LaTeX2e kernel, the most notable of which is that in current
% LaTeX2e releases, the ordering of single and double column floats is not
% guaranteed to be preserved. Thus, an unpatched LaTeX2e can allow a
% single column figure to be placed prior to an earlier double column
% figure.
% Be aware that LaTeX2e kernels dated 2015 and later have fixltx2e.sty's
% corrections already built into the system in which case a warning will
% be issued if an attempt is made to load fixltx2e.sty as it is no longer
% needed.
% The latest version and documentation can be found at:
% http://www.ctan.org/pkg/fixltx2e


%\usepackage{stfloats}
% stfloats.sty was written by Sigitas Tolusis. This package gives LaTeX2e
% the ability to do double column floats at the bottom of the page as well
% as the top. (e.g., "\begin{figure*}[!b]" is not normally possible in
% LaTeX2e). It also provides a command:
%\fnbelowfloat
% to enable the placement of footnotes below bottom floats (the standard
% LaTeX2e kernel puts them above bottom floats). This is an invasive package
% which rewrites many portions of the LaTeX2e float routines. It may not work
% with other packages that modify the LaTeX2e float routines. The latest
% version and documentation can be obtained at:
% http://www.ctan.org/pkg/stfloats
% Do not use the stfloats baselinefloat ability as the IEEE does not allow
% \baselineskip to stretch. Authors submitting work to the IEEE should note
% that the IEEE rarely uses double column equations and that authors should try
% to avoid such use. Do not be tempted to use the cuted.sty or midfloat.sty
% packages (also by Sigitas Tolusis) as the IEEE does not format its papers in
% such ways.
% Do not attempt to use stfloats with fixltx2e as they are incompatible.
% Instead, use Morten Hogholm'a dblfloatfix which combines the features
% of both fixltx2e and stfloats:
%
% dblfloatfix}
% The latest version can be found at:
% http://www.ctan.org/pkg/dblfloatfix




% *** PDF, URL AND HYPERLINK PACKAGES ***
%
\usepackage{url}
% url.sty was written by Donald Arseneau. It provides better support for
% handling and breaking URLs. url.sty is already installed on most LaTeX
% systems. The latest version and documentation can be obtained at:
% http://www.ctan.org/pkg/url
% Basically, \url{my_url_here}.

% For special characters like ä,ö,...
\usepackage[utf8x]{inputenc}


% *** Do not adjust lengths that control margins, column widths, etc. ***
% *** Do not use packages that alter fonts (such as pslatex).         ***
% There should be no need to do such things with IEEEtran.cls V1.6 and later.
% (Unless specifically asked to do so by the journal or conference you plan
% to submit to, of course. )


% correct bad hyphenation here
\hyphenation{op-tical net-works semi-conduc-tor}


\begin{document}
%
% paper title
% Titles are generally capitalized except for words such as a, an, and, as,
% at, but, by, for, in, nor, of, on, or, the, to and up, which are usually
% not capitalized unless they are the first or last word of the title.
% Linebreaks \\ can be used within to get better formatting as desired.
% Do not put math or special symbols in the title.
\title{Enabling Ambient Backscatter Using Low-cost Software-defined-radio}


% author names and affiliations
% use a multiple column layout for up to three different
% affiliations
\author{\IEEEauthorblockN{Maximilian Stiefel}
\IEEEauthorblockA{Master Programme Embdedded Systems\\
Uppsala University\\
maximilian.stiefel.8233@student.uu.se}
\and
\IEEEauthorblockN{Elmar van Rijnswou}
\IEEEauthorblockA{Master Programme Embdedded Systems\\
Uppsala University\\
elmar.vanrijnswou.9818@student.uu.se}
\and
\IEEEauthorblockN{Carlos Pérez-Penichet}
\IEEEauthorblockA{Communication Research Group\\
Uppsala University\\
carlos.penichet@it.uu.se}
\and
\IEEEauthorblockN{Ambuj Varshney}
\IEEEauthorblockA{Communication Research Group\\
Uppsala University\\
ambuj.varshney@it.uu.se}
\and
\IEEEauthorblockN{Christian Rohner}
\IEEEauthorblockA{Communication Research Group\\
Uppsala University\\
christian.rohner@it.uu.se}
\and
\IEEEauthorblockN{Thiemo Voigt}
\IEEEauthorblockA{Communication Research Group\\
Uppsala University\\
thiemo.voigt@it.uu.se}}

% conference papers do not typically use \thanks and this command
% is locked out in conference mode. If really needed, such as for
% the acknowledgment of grants, issue a \IEEEoverridecommandlockouts
% after \documentclass

% for over three affiliations, or if they all won't fit within the width
% of the page, use this alternative format:
% 
%\author{\IEEEauthorblockN{Michael Shell\IEEEauthorrefmark{1},
%Homer Simpson\IEEEauthorrefmark{2},
%James Kirk\IEEEauthorrefmark{3}, 
%Montgomery Scott\IEEEauthorrefmark{3} and
%Eldon Tyrell\IEEEauthorrefmark{4}}
%\IEEEauthorblockA{\IEEEauthorrefmark{1}School of Electrical and Computer Engineering\\
%Georgia Institute of Technology,
%Atlanta, Georgia 30332--0250\\ Email: see http://www.michaelshell.org/contact.html}
%\IEEEauthorblockA{\IEEEauthorrefmark{2}Twentieth Century Fox, Springfield, USA\\
%Email: homer@thesimpsons.com}
%\IEEEauthorblockA{\IEEEauthorrefmark{3}Starfleet Academy, San Francisco, California 96678-2391\\
%Telephone: (800) 555--1212, Fax: (888) 555--1212}
%\IEEEauthorblockA{\IEEEauthorrefmark{4}Tyrell Inc., 123 Replicant Street, Los Angeles, California 90210--4321}}




% use for special paper notices
%\IEEEspecialpapernotice{(Invited Paper)}




% make the title area
\maketitle

% As a general rule, do not put math, special symbols or citations
% in the abstract
\begin{abstract}
Backscatter communication is attractive for energy-constrained devices due to its very low power requirements. Ambient backscatter takes this aspect to the limit by leveraging existing radio frequency signals for the purpose of communication without the need for generating energy expensive carrier signal. In this paper we investigate the use of ambient television broadcast signals for communication. As opposed to state-of-the-art restricted to operations under conditions of strong signal strength, we demonstrate a low cost software defined radio as receiver 
enables operations even in conditions when ambient signals are  weak in strength. We build the system using low-cost off-the-shelf microcontroller, and RTLSDR software-defined radio receiver. We also conduct survey of signal strength of TV broadcast in a mid-sized swedish city, and observe that our system can operate in most parts of the city.
\end{abstract}

% no keywords




% For peer review papers, you can put extra information on the cover
% page as needed:
% \ifCLASSOPTIONpeerreview
% \begin{center} \bfseries EDICS Category: 3-BBND \end{center}
% \fi
%
% For peerreview papers, this IEEEtran command inserts a page break and
% creates the second title. It will be ignored for other modes.
\IEEEpeerreviewmaketitle



\section{Introduction}
% no \IEEEPARstart
% You must have at least 2 lines in the paragraph with the drop letter
% (should never be an issue)

\hfill April 7, 2017

% An example of a floating figure using the graphicx package.
% Note that \label must occur AFTER (or within) \caption.
% For figures, \caption should occur after the \includegraphics.
% Note that IEEEtran v1.7 and later has special internal code that
% is designed to preserve the operation of \label within \caption
% even when the captionsoff option is in effect. However, because
% of issues like this, it may be the safest practice to put all your
% \label just after \caption rather than within \caption{}.
%
% Reminder: the "draftcls" or "draftclsnofoot", not "draft", class
% option should be used if it is desired that the figures are to be
% displayed while in draft mode.
%
%\begin{figure}[!t]
%\centering
%\includegraphics[width=2.5in]{myfigure}
% where an .eps filename suffix will be assumed under latex, 
% and a .pdf suffix will be assumed for pdflatex; or what has been declared
% via \DeclareGraphicsExtensions.
%\caption{Simulation results for the network.}
%\label{fig_sim}
%\end{figure}

% Note that the IEEE typically puts floats only at the top, even when this
% results in a large percentage of a column being occupied by floats.


% An example of a double column floating figure using two subfigures.
% (The subfig.sty package must be loaded for this to work.)
% The subfigure \label commands are set within each subfloat command,
% and the \label for the overall figure must come after \caption.
% \hfil is used as a separator to get equal spacing.
% Watch out that the combined width of all the subfigures on a 
% line do not exceed the text width or a line break will occur.
%
%\begin{figure*}[!t]
%\centering
%\subfloat[Case I]{\includegraphics[width=2.5in]{box}%
%\label{fig_first_case}}
%\hfil
%\subfloat[Case II]{\includegraphics[width=2.5in]{box}%
%\label{fig_second_case}}
%\caption{Simulation results for the network.}
%\label{fig_sim}
%\end{figure*}
%
% Note that often IEEE papers with subfigures do not employ subfigure
% captions (using the optional argument to \subfloat[]), but instead will
% reference/describe all of them (a), (b), etc., within the main caption.
% Be aware that for subfig.sty to generate the (a), (b), etc., subfigure
% labels, the optional argument to \subfloat must be present. If a
% subcaption is not desired, just leave its contents blank,
% e.g., \subfloat[].


% An example of a floating table. Note that, for IEEE style tables, the
% \caption command should come BEFORE the table and, given that table
% captions serve much like titles, are usually capitalized except for words
% such as a, an, and, as, at, but, by, for, in, nor, of, on, or, the, to
% and up, which are usually not capitalized unless they are the first or
% last word of the caption. Table text will default to \footnotesize as
% the IEEE normally uses this smaller font for tables.
% The \label must come after \caption as always.
%
%\begin{table}[!t]
%% increase table row spacing, adjust to taste
%\renewcommand{\arraystretch}{1.3}
% if using array.sty, it might be a good idea to tweak the value of
% \extrarowheight as needed to properly center the text within the cells
%\caption{An Example of a Table}
%\label{table_example}
%\centering
%% Some packages, such as MDW tools, offer better commands for making tables
%% than the plain LaTeX2e tabular which is used here.
%\begin{tabular}{|c||c|}
%\hline
%One & Two\\
%\hline
%Three & Four\\
%\hline
%\end{tabular}
%\end{table}


% Note that the IEEE does not put floats in the very first column
% - or typically anywhere on the first page for that matter. Also,
% in-text middle ("here") positioning is typically not used, but it
% is allowed and encouraged for Computer Society conferences (but
% not Computer Society journals). Most IEEE journals/conferences use
% top floats exclusively. 
% Note that, LaTeX2e, unlike IEEE journals/conferences, places
% footnotes above bottom floats. This can be corrected via the
% \fnbelowfloat command of the stfloats package.

\section{Background}

\subsection{RTL2832U}
\label{sub:rtl2832}
\begin{figure}[h]
\centering
\includegraphics[width=0.5\columnwidth]{./fig/rtlsdr.jpg}
\caption{\textit{RTL-SDR} hardware with the DVB-T I/Q demodulator \textit{Raeltek RTL2832U} (left IC) and the tuner with integrated LNA \textit{Rafael Micro R820T/2} (right IC).}
\label{fig:receiver_arch}
\end{figure}
The Raeltek RTL2832U is a terrestrial digital video broadcast (DVB-T) chip, which is the basis of a very low-cost software defined radio: The RTL-SDR. One feature of this chip is, that it supports USB 2.0. Ham radio enthusiasts combined their efforts to enable accessing the raw I/Q data. With their software driver (basically the \textit{librtlsdr}, cf. \cite{steve-m_librtlsdr}) DVB-T sticks, which are based on the RTL2832, can be converted into a wideband SDR. This sort of hardware has been in the range of a few thousands USD a few years ago. An RTL2832U based DVB-T receiver can be bought with antenna for less than 10 USD. 
What is a software defined radio? Traditionally basic radio components like mixers or filters have been realized in hardware. Nowadays as soon as one works in the baseband frequency the coputational resources of a personal computer are sufficient to realize these components with signal processing done in software.
An RTL-SDR does not only consist of the RTL2832U. The second essential component is the tuner, which mixes down the radio frequency (RF) to a intermediate frequency (IF). In the case of the used hardware this tuner component was represented by a \textit{Rafael Micro R820T/2}. It subsumes a low-noise amplifier (LNA) and filters. The image frequency is rejected automatically with 65 dBc and the noise figure is with 3.5 dB quite low. It alows a frequency range from 42 to 1002 MHz (cf. \cite{rafael_r820t_2011}). 
The I/Q demodulator, also shown in figure \ref{fig:receiver_arch}, is the most important part of the \textit{RTL2832U} for our work. The received signal can be intepreted as
\begin{equation}
	s_{IF}(t)=I(t) \cdot cos(\omega_{0}t) + Q(t) \cdot sin(\omega_{0}t)
\end{equation}     
This is multiplied with a cosine (and a sine as well) from a local oscillator.
\begin{multline}
        s_{IF}(t) \cdot s_L(t) = I(t) \cdot cos(\omega_{0}t) \cdot cos(\omega_{0}t)\\+ Q(t) \cdot sin(\omega_{0}t) \cdot cos(\omega_{0}t)
\end{multline}
With \ensuremath{2\,cos(a)cos(b)=cos(a-b)+cos(a+b)} and \ensuremath{2\,sin(a)cos(b)=sin(a+b)+sin(a-b)} follows
\begin{multline}
        2 \cdot s_{IF}(t) \cdot s_L(t) = I(t) \cdot \bigl[1+cos(2\omega_{0}t)\bigr]\\+ Q(t) \cdot \bigl[sin(2\omega_0t)+sin(0)\bigr].
\end{multline}
One can see, that the interesting inphase part in this case is represented by a DC value after mixing. With a low-pass filter this DC value can be seperated from the undesired rest. Similar things are hapening with the quadrature part when mixing with a sine, which is indeed done to get the quadrature part.  
\section{Design}

\subsection{Transmitter}
\begin{figure}[h]
\centering
\includegraphics[width=0.8\columnwidth]{./fig/s11v2}
\caption{Input reflection coefficient \ensuremath{\Gamma_{in}} also known as \ensuremath{S_{\text{11}}} of the selfmade backsacttering ground plane antenna in dB.}
\label{fig:s11}
\end{figure}
A backscattering antenna is the most important part of backscattering sender. Our antenna is optimized for usage with the television signal, which has its center frequency at 626 MHz. It is a ground plane antenna, which has been built up on a big PCB. This rectangular PCB, with dimensions of 200 mm x 300 mm, is covered with a copper layer on one side and left blank on the other. A hole in the midle of the PCB serves for connecting a thicker wire to a 50 \ensuremath{\Omega} microstrip transmission line. The wire is the vertical antenna element has a length of 350 mm, which is roughly 0.73 \ensuremath{\lambda}. It is soldered to the transmission line. The transmission line has been calculated for 626 MHz using FR4 substrate with a thickness of 1.6 mm. According to microstrip line theory the length of the line changes the phase and the width of the line changes the impedance (cf. chapter 3 in \cite{pozar_microwave_2011}). The design equations given in \cite{pozar_microwave_2011} are implemented in various programs e.g. KiCAD, which has been used for solving the design equations. The given parameters lead to a microstrip line width of 3 mm. An open-stub, attached directly at the puncture where the vertical antenna element is attached to the transmission line, is used for fine tuning of the input impedance. The stub length has been adjusted manually (shortened) while measuring the reflection coefficient with a vector network analyzer (VNA). Open-stub matching is discussed in chapter 5 in \cite{pozar_microwave_2011}. Furthermore the resonance frequency has been tuned by cutting the vertical antenna element piece by piece while measuring with the VNA. At some point the \ensuremath{S_{\text{11}}} had the minimum quite close to where we wanted to have it (cf. \ref{fig:s11}). In the plot one can also see, that the antenna is quite narrowband. For backscattering this is not necessarily a disadvantage, since everything, which is within the operating frequency of the antenna is backscattered. So with a narrowband antenna the effect, that signals are backscattered, which are not targeted, is reduced. This statement has to be qualified as the antenna also has multiple resonance frequencies in multiple bands. With our approach we were able to achieve a input resistance at 634 MHz of \ensuremath{Z_{in}=(52.2 + j 4.3) \Omega}. At 626 MHz \ensuremath{S_{\text{11}}} is still approximately -17 dB.                   
Besides the antenna the transmitter consists out of a \textit{MSP430}, which acts as subscriber, that sends data to the receiver. This transmitter is connected with one I/O pin and ground to a rf switch. The switch can change between matched (50 \ensuremath{\Omega}) and open. The software now steers this switch with a two 2 MHz rectangular timer signal for a 1 and is just leavs it open for a 0. Therefore the result is a shift of the television signal by 2 MHz for a 1 and no shift for a 0. Hence a classical binary amplitude shift keying is the implemented modulation technique.    

\subsection{Receiver}
\begin{figure}[h]
\centering
\includegraphics[width=\columnwidth]{./fig/receiver_arch}
\caption{Receiver architecture from a system point of view. Sinal flow is from the left to the right.}
\label{fig:receiver_arch}
\end{figure}
The receiver was the most work intensive part of the design. C++ has been used to implement the receiver. We provide the code publicly available under \url{https://github.com/s3xm3x/backscatterBASKReceiver}. In figure \ref{fig:receiver_arch} one can see the architecture of the receiver from an abstarct point of view. A highly sophisticated bus system has been developed to exchange data between the different components of the system. The signal flow is from the left to the right. \textit{Librtlsdr} (cf. \cite{steve-m_librtlsdr}), which is based on \textit{libusb}, is used to transfer the data from the TV stick into our program. As described under \ref{sub:rtl2832}, the RTL2832 mixes down the high frequency to a intermediate frequency (both values are in SW adjustable). Subsequently the data is mixed with sine and cosine, low-pass filtered and finally transfered in the digital world. This data is available as two 8 Bit values. It is a well-known fact, that this technique, which is commonly known as I/Q demodulation, results in a real and an imagenery part. With this two values phase and absolute value of the data can be determined for every sample. So every sample consists out of a real and and imaginary part. And can be seen as:
\begin{equation}
	I+j\,Q = abs(I,Q) \cdot e^{ang(I,Q)} 
\end{equation} 
So the first block, which is entitled with RTL2832U, provides the interface to the TV dongle. This block is represented as one object. It can set the frequency \ensuremath{f_{tuned}}, where the receiver is listening. Also it can set the sampling frequency \ensuremath{f_{samp}} and the analog gain of the tuner. The raw samples are pushed to the bus. 
Demodulator is the block called, which is responsible to convert the sampled signals into something, which looks like ones and zeroes. Thereforre the real and imgainary values first have to be deinterleaved out of the sample message. Now they are represented as integer values between 0 and 255 (8 bit). So one converts them to floats, whereas 127.5 equals 0. After that the values have to be downsampled. Usually we were sampling with 250 kHz. This is still much more than we need assuming an additive white gausian noise (AWGN) channel with a \ensuremath{S/N = 10\,dB} (cf. equation \ref{eq:awgn}). So we decided to reduce the sampling frequency to 25 kHz.
\begin{equation}
	\label{eq:awgn}
	C=B \cdot log_2(1+\frac{S}{N})= 25\,kHz \cdot log_2(11) \approx 86.5\,kbit/s
\end{equation}     
To do this it is important to use an anti-aliasing filter, as the Shannon-Nyquist theorem has to be satisfied: 
\begin{equation}
	\label{eq:nyquist}
	f_{samp} \geq 2 \cdot f_{a}
\end{equation}
\ensuremath{f_{a}} is the highest frequency in the signal. 
A FIR filter naivly implemented with floats has been used to achieve this. 10 kHz is the cut-off frequency of the first filter. All filters have been constructed with the internal filter design tool of GNU radio. After downsampling another filter is used to suppress noise. This filter has a cut-off frequency of 3 kHz. The last demodulation step is to take the signal, rectify it (cf. equation \ref{eq:rectify}) and decide with a software-defined Schmitt trigger whether a sample is a 1 or a 0. 
\begin{equation}
	\label{eq:rectify}
	abs(I,Q) = \sqrt{I^2 +  Q^2}
\end{equation} 
These samples are broadcasted on the bus. A registered listener receives the processed samples. This registered listener is the decoder, which is represented as an object as well. After the decoder received the samples it decides when a frame is sent or when the channel is idle. On the basis of a defined threshold the decoder determines wether a certain sample pattern of ones and zeros is a 1 bit or a 0 bit. The baudrate and the sampling rate lead to a certain length of one bit. For example a sample rate \ensuremath{f_{sampling}} of 25 kHz combinded with a baudrate of 1 kbit/s leads to 25 samples/bit. The decoder also includes a function to correlate the received bitstream with an expected pattern. Hence it is able to determine the bit ro error ratio (BER). 
Furthermore the code we have written so far subsumes a simulator to simulate ideal data coming out of the demodulator to test the algorithms behind the decoder. Also a simulator for the RTL2832U has been written to replay recorded samples, which facialiates debugging as well.    
Another handy utility, which has been written in Octave, is a tool (an oscilloscope) to look at the data, which is written into files by the demodulator. An example of a recorded data stream with 25 kS/s can be seen in figure \ref{fig:osci}. 
\begin{figure}[h]
\centering
\includegraphics[width=\columnwidth]{./fig/frame_and_idle}
\caption{Transmission of 101010.. with 1 kbit/s and idle state. Data from the Octave oscilloscope. Amplitude is quite low with an average of 12 \% of the maximum, which is 127.5.}
\label{fig:frame}
\end{figure}

\section{Evaluation}

\subsection{Signal Strength Variation within the City}
To observe the signal strength variation within the city the \textit{RTL-SDR} has been used as well. 

\subsection{Signal Strength Variation over Time}

\subsection{Communication Performance}

\section{Discussion}

% conference papers do not normally have an appendix


% use section* for acknowledgment
\section*{Acknowledgment}


The authors would like to thank...





% trigger a \newpage just before the given reference
% number - used to balance the columns on the last page
% adjust value as needed - may need to be readjusted if
% the document is modified later
%\IEEEtriggeratref{8}
% The "triggered" command can be changed if desired:
%\IEEEtriggercmd{\enlargethispage{-5in}}

% references section

% can use a bibliography generated by BibTeX as a .bbl file
% BibTeX documentation can be easily obtained at:
% http://mirror.ctan.org/biblio/bibtex/contrib/doc/
% The IEEEtran BibTeX style support page is at:
% http://www.michaelshell.org/tex/ieeetran/bibtex/
\bibliography{paper}
\bibliographystyle{IEEEtran}
% argument is your BibTeX string definitions and bibliography database(s)
%
% <OR> manually copy in the resultant .bbl file
% set second argument of \begin to the number of references
% (used to reserve space for the reference number labels box)
%\begin{thebibliography}{1}
%
%\bibitem{github:librtlsdr}
%Steve Markgraf et. al. \emph{librtlsdr}. \url{https://github.com/steve-m/librtlsdr}. Github: 2012-2017.
%
%\bibitem{rafael:r820t}
%\emph{High Performance Low Power Advanced Digital TV Silicon Tuner}. R820T. Rev 1.2. Rafael Micro. November 2011.
%
%\bibitem{book:pozar}
%D. M. Pozar. \emph{Microwave Engineering}. Fourth Edition. John Wiley \& Sons Inc. 111 River Street, Hoboken. 2012
%
%\end{thebibliography}




% that's all folks
\end{document}


