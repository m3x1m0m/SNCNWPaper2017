Consider an unmodulated carrier wave impinging on a backscatter tag's
antenna. The signal observed by the receiver is:
\begin{equation}
				S_r(t) = S_{rc}(t) + \sigma B(t)S_{bc}(t)
				\label{eq:signals}
\end{equation}
where $S_{rc}(t)$ is the signal coming directly from the carrier
generator and $S_{bc}(t)$ is the signal from the carrier generator that
reaches the backscatter tag. $B(t)$ is either zero or one and represents
the instantaneous state of the backscatter tag: absorbing or reflecting,
respectively.

Equation~(\ref{eq:signals}) reveals an important issue for backscatter
communication systems: self-interference. The carrier signal $S_{rc}(t)$
interferes at the receiver with the data-carrying signal from the tag,
$\sigma B(t)S_{bc}(t)$. 

Recent work on generating backscatter transmissions has avoided
self-interference through \textit{frequency-shifted}
backscatter~\cite{kellogg_passive_2016,wang_fm_2017,varshney2016lorea}.
The tags modulate their antenna in such a way that their transmissions
occur at a certain frequency offset from the carrier signal thus
allowing the receiver to avoid interference from the carrier by tuning
to the offset frequency where the tag is transmitting.

Consider the case when $B(t)$ periodically
alternates between its two states at a frequency $\Delta f$ while an
unmodulated carrier of frequency $f_c$ reaches the backscatter tag:

\begin{equation}
    2\sin(f_c t) \sin(\Delta f t) = \cos[(f_c + \Delta f)t] - \cos[(f_c - \Delta f) t]
    \label{eq:mixing}
\end{equation}

If we focus on only one of the two generated images, or employ single
sideband backscatter~\cite{interscatter}, the data transmission can now
be received at frequency $f_d = f_c + \Delta f$.

