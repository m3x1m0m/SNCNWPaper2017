Backscatter communication enables ultra-low
power wireless transmissions, and is attractive
for networking devices which operate 
on harvested energy. Ambient backscatter
takes the concept further, by leveraging
ambient wireless signals like
television signals, as both the source of power
and carrier signal. Existing state-of-the-art ambient backscatter
systems demonstrate ability to achieve tag-to-tag communication 
under conditions when the
strength of TV signals  sufficiently high (-30 dBm).  In this paper, we present our preliminary work
which demonstrates the possibility to backscatter and communicate even when the ambient television signals
are significantly weaker. The key to achieving this is to leverage a low-cost
software defined radio receiver (RTL-SDR) to receive backscattered transmissions. Our results 
demonstrate that the  television signals are strong 
enough in most parts of a mid-sized swedish city for our system
to operate, and communicate, a significant improvement over the
state-of-the-art which are
restricted to operate only when the tags are close to the TV towers.
