Backscatter communication enables wireless transmissions
at an energy consumption several orders of magnitude
lower than traditional radios. Backscatter achieves ultra-low
power wireless transmissions by reflecting or absorbing ambient
wireless signals, which consumes only 
\SI{}{\micro\watt}s of power~\cite{liu_ambient_2013}. 
As a consequence backscatter
communication is emerging as the mechanism of choice to network
devices operating on harvested energy.  
Over the past few years there has been significant progress
to make backscatter a viable mechanism to network Internet of
Things~(IoT) devices. Traditional backscatter systems, like RFID readers, required a
dedicated device to generate the necessary carrier signal
reflected by the tags. On the other hand, state-of-the art systems does away
with the need for a dedicated device to generate carrier signals.  Recent
backscatter systems leverage the already deployed infrastructure such as
WiFi~\cite{hitchhike,kellogg2015wi} or TV
towers~\cite{liu_ambient_2013,parks_turbocharging_2014} to generate the carrier
signal. %~\cite{varshney2016lorea,iyer2016inter}, or ambient WiFi.

Recent backscatter systems demonstrate the ability to leverage
existing signals, such as TV, as the source of both carrier and energy. For
example, Liu et al. present a proof-of-concept system that reflects
ambient TV signals and enables tag-to-tag communication up to almost a
meter. Parks et al. further improve the communication range to several
meters by using analog coding~\cite{parks_turbocharging_2014}. While
these systems can enable many applications, they are severely restricted
to operate in the vicinity of television towers where ambient signal
are sufficiently strong~(approx
\SI{-30}{dBm}). This is primarily due to the poor sensitivity of the receivers
employed on these devices, which together with weak backscatter reflections 
severely limits the operating range from the tower.
Furthermore, TV signals are known to vary greatly in
strength~\cite{wang_fm_2017} both over space, and in time,  which further aggravates the
problem of the limited range of ambient backscatter systems.

On the other hand, Software Defined Radios~(SDRs) are powerful devices, and also
have significant processing abilities. These devices offer high
sensitivity levels, as compared to the receivers employed on typical 
ambient backscatter tags. Thus, SDRs might help to
significantly improve the communication range, and also coverage 
area to receive ambient backscatter transmissions. Moreover the I/Q demodulator ICs, which represent the heart of an SDR system are very cheap as they are used a lot in digital communication. We are confident to be able to accelerate the research in the backscattering field, since we introduce a backscattering equipement, which is easily affordable by any student.   

In this paper we explore the following questions: Can we leverage an
SDR-based receiver to receive ambient backscatter transmissions?, and, Does the  
relatively high sensitivity of SDR receivers improve the
range and coverage of ambient backscatter systems? A positive answer to the above questions would provide a
flexible and low-cost experimentation platform to the wider research community to explore ambient backscatter
systems.

\fakepar{Contributions}  In this paper, we make the following novel contributions:

\begin{itemize}
		

				\item We design the first system that leverages a  low-cost 
					  SDR to receive ambient-backscatter transmissions. 
				\item  We demonstrate ambient backscatter using TV signals
					   to be feasible in wide parts of a city. The range represents a significant improvement
					   over the state-of-the-art.
				
\end{itemize}

The paper proceeds as follows. First, we discuss relevant background information of backscatter communication,
SDR and reception process. Next, we present the design of the transmitter and the receiver employed in this
paper. We then present initial results which evaluates our system. Finally, we present concluding remarks.
