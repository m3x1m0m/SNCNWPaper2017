Backscatter enables wireless communication at orders of magnitude lower
energy cost than traditional radios. A backscatter transmitter operates
by either reflecting or absorbing existing wireless signals, which only
incurrs in very low power consumption in the order of
micro-Watts~\cite{liu_ambient_2013}. As a consequence backscatter
communication is emerging as the mechanism of choice to network
battery-free devices.  Over the past few years significant progress has
been made to make backscatter communication practical for Internet of
Things~(IoT) devices. Early backscatter systems, like RFID, required a
dedicated device to generate the necessary carrier to be backscattered.
State-of-the-art systems, however, reflect ambient WiFi or television
signals~\cite{liu_ambient_2013} instead, which does away with the need for
dedicated infrastructure. 

Recent ambient backscatter systems demonstrate tag-to-tag communication
by leveraging TV signals as a source of both carrier and energy. For
example, Liu et al. present a proof-of-concept system that reflects
ambient TV signals and enables tag-to-tag communication up to almost a
meter. Parks et al. further improve the communication range to several
meters by using analog coding~\cite{parks_turbocharging_2014}. While
these systems can enable many applications, they are also restricted to
operate only in environments where TV signals are fairly strong~(approx
\SI{-30}{dBm}) due to the low sensitivity of the receiver used in these
tags. On the other hand, TV signals are known to vary greatly in
strength~\cite{wang_fm_2017} over space and time, making existing
systems very restricted in their coverage.

On the other hand, Software Defined Radios~(SDRs) are powerful devices
that have significant processing abilities and offer relatively high
receive sensitivity levels. The high sensitivity levels of the SDRs,
compared to the typical ambient backscatter receiver tag, could
significantly help improve the communication range and coverage when
receiving ambient backscatter transmissions. More importantly, SDRs
offer incomparable flexibility due to their ability to be reprogrammed.
This flexibility offers the opportunity for ample experimentation, which
is welcomed in this sort of emerging technology.

In this paper we explore the following questions: Can we create an
SDR-based receiver for ambient backscatter transmissions? Can we
leverage the relatively high sensitivity of such a receiver to improve
range and coverage of ambient backscatter systems? Can we provide a
flexible and low-cost experimentation platform for ambient backscatter
research?

\fakepar{Contributions} The contributions of this work are twofold:
\begin{itemize}
				\item We perform a survey of the signal strength of TV
								broandcast signals.

				\item We build a low-cost SDR-based receiver capable of
								receiving data encoded in backscatterer TV signals.
\end{itemize}

